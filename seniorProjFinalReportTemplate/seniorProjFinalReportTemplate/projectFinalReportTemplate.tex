% =======================================
% Computer Science Capstone Report Template
% =======================================
\documentclass[titlepage]{article}
% Add packages here as needed
\usepackage[margin=1in]{geometry} 
\usepackage{amsmath,amsthm,amssymb}

\begin{document}

\title{Final Report\\Title of Project Capstone Project} 
\author{Your Name Here\\\\Advisor: Your Faculty Advisor's Name Here\\Semester: Semester of Completed Project} 
\date{\today}
\maketitle

\abstract{The Computer Science capstone report is a 2000---3000 word essay/paper that addresses the areas/topics described below. This template is intended to provide guidance regarding the structure and content of the report. The report itself must be written in paragraph/narrative format. This template document is structured specifically for reporting senior project capstones, more specifically those that focus on implementing a software product. You should be able to ``mine'' your capstone proposal for some part of the information in the final report, but the report must be written in past tense (because it's done now) and must reflect any changes from the original proposal. Please see the \textit{Computer Science Major's Capstone Requirements} document for more detailed information and explanation of the ideas discussed in this template.}

\section{Introduction}
In the Introduction, you will explain what you did in broad terms. In the introduction, give an overview of the project you completed.

In the Introduction, you will explain what you did and what the output was. That is, what was the proposed project, and what did you produce? You should explain why you wanted to do this project: talk about your interest in and connection to the project. You may include ideas such as what gave you the initial idea for the project, how the project meets a need, and/or how the project provides an improved solution to a problem. The precise details that you choose to include will vary greatly depending on the context of the work that you completed. This section serves as your ``problem statement,'' in which you give a clear and succinct statement of what your intention/idea was with your project, or the problem you were trying to solve.

As with the proposal template, I have included a sample \texttt{.bib} file and some meaningless examples of \LaTeX{ }markups for cross-references \cite{pholdee_hybrid_2017}, in case you have need of references. The need for references is less likely for an internship report, but the information is included here for completeness. Just for the sake of a second example, here is another reference citation \cite{hadka_large-scale_2015}. If desired, you may change citation and reference format by changing the bibliography style in the template. The current style is ``acm'' and is appropriate for Computer Science topics. If you change formats, be sure to check with your faculty advisor in advance to make sure the style is appropriate for Computer Science.

\section{Background}
The Background section will discuss the process by which you developed and prepared for your project work. This section includes the subsections Preparation and Practice, as described below.

\subsection{Preparation}
In this section, you will discuss how you prepared to undertake this project. This section often includes reference to specific Computer Science courses that provided the background and fundamental skills to support your project work. You may also include other experiences (\textit{e.g.,} previous independent projects, jobs) that contributed to your preparation.

\subsection{Practice}
Any capstone experience must both bring together your academic learning \textit{and} go beyond that academic preparation. You will address those issues in this section. How did the capstone synthesize and surpass your academic experiences? You should also include particular professional practice skills that you cultivated with this experience, specifically considering non-technical skills.

\section{Methods}
This section will provide details about the technical tools and methods you used to produce your project---the building blocks. The section can be organized in different ways that will vary considerably from one project to another. Information that should be detailed in this section includes:
\begin{itemize}
\item Specific technical tools used (languages, libraries, toolkits, platforms, \textit{etc.}).
\item Design and implementation process.
\item Development methodology that you used.
\item Coding standards you used in your project.
\item Management tools you used for the project, including communications tools, organization, and version control.
\end{itemize}

The Methods section is where you put all the gory technical details. Focus some attention on the tools/skills that were new to you in the project work.

\section{Results}
This section addresses in more detail what you produced for the capstone. Where the Methods section describes how you tackled the problem, the Results section shows the final outcome. 

As appropriate, you can include code segments or screen captures to illustrate your technical solution and the end product. This section is the equivalent of the live demo during a project presentation. You should show the structure and flow of the user experience with your software, as completely as possible.
 
\section{Conclusion} 
To conclude your report, bring all the threads of your discussion together to reflect upon how the experience made you a better Computer Scientist. As part of this reflection, provide advice for other students who are preparing for senior projects, with particular focus on things you would have liked to know before beginning your senior project experience.
 	
\bibliography{sampleBibFile}
\bibliographystyle{acm}

\end{document}
